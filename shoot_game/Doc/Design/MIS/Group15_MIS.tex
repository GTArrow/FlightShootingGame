\documentclass[12,english]{article}
\usepackage[letterpaper, portrait, margin=1in]{geometry}

\usepackage{amsmath}
\usepackage[T1]{fontenc}
\usepackage{babel}
\usepackage{textcomp}
\usepackage{titlesec}
\setcounter{secnumdepth}{4}
\usepackage{hyperref}
\usepackage{xcolor}
\usepackage{booktabs}
\usepackage{placeins}
\usepackage{multirow}
\usepackage{color}
\usepackage{ulem}


\titleformat{\paragraph}
{\normalfont\normalsize\bfseries}{\theparagraph}{1em}{}
\titlespacing*{\paragraph}
{0pt}{3.25ex plus 1ex minus .2ex}{1.5ex plus .2ex}

\title{Module Internal Specfication for Group 15 - FlightShootingGame}
\author{Yijun Chen\\
\texttt{ (cheny161)}
\and
Tianxing Li\\
\texttt{(Lit20)}
\and
Zefeng Wang\\
\texttt{(wangz217)}
}

\date{}


\begin{document}
\maketitle
\newpage
\tableofcontents
\newpage

\section{Module Hierarchy}
\begin{table}[!htbp]
        \begin{tabular}{ll}
        \toprule
        Level 1 & Level 2 \\
        \midrule
        \sout{Hardware Hiding Module} & \\
         \midrule
        Behaviour Hiding Module & Boss Module \\%1
        & Bullet Module\\%2
        & Smart\_Bullet Module\\%3
        & Enemy Module, Player Module\\%4, 5
        & Gift Module\\%6
		& Music Module \\%7
		 \midrule
        Software Decision Module & \\
        \bottomrule
        \end{tabular}
        \caption{Module Hierarchy}
        % Colour for the rulings in tables:
        \makeatletter
           \def\rulecolor#1#{\CT@arc{#1}}
           \def\CT@arc#1#2{%
           \ifdim\baselineskip=\z@\noalign\fi
           {\gdef\CT@arc@{\color#1{#2}}}}
           \let\CT@arc@\relax
          \rulecolor{black!50}
        \makeatother
        \label{Table 1}
        \end{table}

\section{MIS of Boss Module}
		\subsection{Interface Syntax}
		\subsubsection{Exported Access Programs}
		\begin{tabular}[pos]{|c|c|c|c|}
			
			\hline
			%	\label
			\textbf{Name}& \textbf{In} & \textbf{Out} & \textbf{Exceptions} \\ \hline
			Constructor & Integer, Integer, Integer & Boss Object & -\\ \hline
			draw & - & GUI & -\\ \hline
			move & - & - & -\\ \hline
			show\_up & - & - & -{\color{red} ObjectIsShowingUp}\\ \hline
			check\_death & integer, integer &
			integer & -\\ \hline
			reset & integer & - & -\\ \hline
			
		\end{tabular}
		
		\subsection{Interface Semantics}
		\subsubsection{State Variables}
		state: int - the state of the object that determine the behaviour of the object.\\
		life: int - the life point of the object.\\
		x: int - the x value of the position of the object.\\
		y: int - the y value of the position of the object.\\
		direction: int - the direction of the object's movement.\\
		
		\subsubsection{Environmental Variables}
		Not Applicable
		
		\subsubsection{Assumptions}
		$0\leq x\leq 115$\\
		$0\leq y\leq 185$\\
		$state \in \{0, 1\}$
		
		\subsubsection{Access Program Semantics}
		draw():
		
		Input: none
		
		Output: graphics on screen.
		
		Exceptions: none.\\
		\\
		move():
		
		Input: none
		
		Transition: get the data from the object and move its position.
		
		Exceptions: none.\\
		\\
		show\_up():
		
		Input: none
		
		Transition: change the state of the object to 1 if it is initially 0.
		
		Output: the transition of the object's state.
		
		Exceptions: \sout{none} {\color{red} ObjectIsShowingUp}\\
		\\
		check\_death(life, point):
		
		Input: life point, score point
		
		Transition: if the life point of the object is 0, then reset it to `life' and then return score `point'.
		
		Output: Transition of object's position.
		
		Exceptions: none.\\
		\\
		reset(life):
		
		Input: life, the life point that is to be set. 
		
		Transition: None
		
		Output: The None.
		
		Exceptions: none.
\section{MIS of Bullet Module}
		\subsection{Interface Syntax}
		\subsubsection{Exported Access Programs}
		\begin{tabular}[pos]{|c|c|c|c|}
			
			\hline
			%	\label
			\textbf{Name}& \textbf{In} & \textbf{Out} & \textbf{Exceptions} \\ \hline
			Constructor & integer, integer, integer, integer, integer, integer, integer & bullet object & -\\ \hline
			move & - & - & -\\ \hline
			draw & - & GUI & -\\ \hline
			reset & integer, integer & - & -\\ \hline

		\end{tabular}
		
		\subsection{Interface Semantics}
		\subsubsection{State Variables}
	    x: int - x value of the object.\\
		y: int - y value of the object.\\
		\subsubsection{Environmental Variables}
		
		\subsubsection{Assumptions}
		Variables should be set before trying to access them
		
		\subsubsection{Access Program Semantics}
		
		move():
		
		Input: none
		
		Transition: change the position of the bullet.
		
		Exception: none\\ 
		\\
		draw():
		
		Input: none
		
		Output: graph on the screen.
		
		Exception: none\\ 
		\\
		reset(x, y):
		
		Transition: change the position of the object to x and y.
		
		Exception: none\\ 
		\\
	
	
	
\section{MIS of Smart\_Bullet Module}
	\subsection{Interface Syntax}
		\subsubsection{Exported Access Programs}
		\begin{table}[!htbp]
		\begin{tabular}{|c|c|c|c|}
			\hline
			Name & In & Out & Exceptions \\ \hline
			constructor & integer, integer, integer, integer, integer, integer & smart\_bullet object & - \\ \hline
			move & - & - & - \\ \hline
			draw & - & GUI & - \\ \hline
			aim & player & - & Invalid Input \\ \hline
			reset & integer, integer &  - & - \\ \hline
		\end{tabular}
	\end{table}
		
	\subsection{Interface Semantics}
		\subsubsection{State Variables}
		x: int - the x value of the object position.\\
		y: int - the y value of the object position.\\
		speed\_x: int - the x component of its speed.\\
		speed\_y: int - the y component of its speed.\\
		
		\subsubsection{Environmental Variables}
		Not applicable.
		\subsubsection{Assumptions}
			All variables are set when the object is constructed. 

		\subsubsection{Access Program Semantics}

        move():

		Input: no inputs
		
		Transition: change the position of the object.
		
		Exception - None\\
		\\
		draw():
		
		Input: none
			
		Output: graph on the screen.
			 
		Exception: none\\
		\\
		aim(player):
			
		Input: player - the player object.
			 
		Transition - updates the direction of the speed by change the value of x\_speed and y\_speed.
			  			 
		Exception - None\\
		\\
		reset(x, y):
		
		Transition: change the position of the object to x and y.
		
		Exception: none\\ 
		\\


\section{MIS of Player Module}
	\subsection{Interface Syntax}
		\subsubsection{Exported Access Programs}
		
	\begin{tabular}[pos]{|c|c|c|c|}
	\hline
	\textbf{Name}& \textbf{In} & \textbf{Out} & \textbf{Exceptions} \\ 
	\hline
	draw & - & GUI & -\\ 
	\hline
	reset & integer & - & -\\
	\hline
	death & - & - & -\\
	\hline
					
	\end{tabular}		
		
	\subsection{Interface Semantics}
		\subsubsection{State Variables}
		x: int - the x value of the object position.\\
		y: int - the y value of the object position.\\
		speed\_x: int - the x component of its speed.\\
		speed\_y: int - the y component of its speed.\\
		\subsubsection{Environmental Variables}
		Not Applicable
		\subsubsection{Assumptions}
All variables are set when the object is constructed.

		\subsubsection{Access Program Semantics}
	    draw():
		
		Input: none
			
		Output: graph on the screen.
			 
		Exception: none\\
		\\
		reset(life):
		
		Input: a integer, life
		
		Transition: change the life value of object to the input value and change the rest to their initial value.
	            	
	   Output: none
	
		Exception: none\\ 
		\\
		death():
		
		Input: none
		
		Transition: change the life value of the object to 0.
	
	    Output: none
	
		Exception: none\\ 
		\\


\section{MIS of Enemy Module}
	\subsection{Interface Syntax}
		\subsubsection{Exported Access Programs}
		\begin{table}[!htbp]
				\begin{tabular}{|c|c|c|c|}
					\hline
					Name & In & Out & Exceptions \\ \hline
					move & - & - & - \\ \hline
					reset & - & - & - \\ \hline
					check\_death & - & integer & -\\ \hline
					not\_collidable & - & - & -\\ \hline
				\end{tabular}
			\end{table}
	\subsection{Interface Semantics}
		\subsubsection{State Variables}
		x: int - the x value of the object position.\\
		y: int - the y value of the object position.\\
		speed: int - the y component of object speed.\\
		\subsubsection{Environmental Variables}
		
		Not Applicable 
		
		\subsubsection{Assumptions}
		
		All variables are set when the object is constructed.

		\subsubsection{Access Program Semantics}
		
		move():
		
		Input: none
		
	  Transition: updates object position
	  
	  Output: none
	  
	  Exception - none \\
	  \\
	  draw():
		
		Input: none
			
		Output: graph on the screen.
			 
		Exception: none\\
		\\
		reset():
		
		Input: none
		
		Transition: set the values of the object to its initial values. 
			
		Output: none
			 
		Exception: none\\
		\\
		check\_death():
		
		Input: none
	
		Output: 0 if the life value is less than 0;
		
		        100 if the life value is not less than 0.
			 
		Exception: none\\
		\\
		not\_collidable():
		
		Input: none
		
		Transition: change the x and y values both to -1000.
	   
	    Output: none

	
\section{MIS of Gift Module}
		\subsection{Interface Syntax}
			\subsubsection{Exported Access Programs}
				\begin{tabular}[pos]{|c|c|c|c|}
					
					\hline
					%	\label
					\textbf{Name}& \textbf{In} & \textbf{Out} & \textbf{Exceptions} \\ \hline
					Pmove & integer & - & Invalid Input\\ \hline
					draw & - & GUI & -\\ \hline
					reset & - & - & -\\ \hline
					
				\end{tabular}
				
		\subsection{Interface Semantics}
			\subsubsection{State Variables}
			X: int - x value of left corner of the gift\\
			Y: int - y value of left corner of the gift\\
			type: int - type of the gift\\
			width: int - width of the gift\\
			height: int - height of the gift\\
			life: int - life of the player
			
			\subsubsection{Environmental Variables}
			Not Applicable
			
			\subsubsection{Assumptions}
			Variables should be set before trying to access them
			
			\subsubsection{Access Program Semantics}
			move(score):
			
			Input: Integer of player score value
			
			Transition: Whenever player score increases 1500, a gift would drop from top and move only in y direction.
			
			Output: none
			
			Exception: none\\
			\\
			draw():
			Input: none
			
			Transition: graph on the screen
			
			Output: none
			Exception: none\\
			\\
			reset():
			
			Input: none
			
			Transition: Change the position of the object to x and y, and change life to 1.
			
			Output: none
			
			Exception: none

			
\section{MIS of Music Module}
		\subsection{Interface Syntax}
			\subsubsection{Exported Access Programs}
				\begin{tabular}[pos]{|c|c|c|c|}
					
					\hline
					%	\label
					\textbf{Name}& \textbf{In} & \textbf{Out} & \textbf{Exceptions} \\ \hline
					landpage\_music & - & - & -\\ \hline
					game\_music & - & - & -\\ \hline
					end\_music & - & - & -\\ \hline
					{\color{red}gift\_music} & {\color{red}-} & {\color{red}-} & {\color{red}-}\\ \hline
					shoot\_music & - & - & -\\ \hline
					damage\_music & - & - & -\\ \hline
					stop\_music & - & - & -\\ \hline
					
				\end{tabular}
				
		\subsection{Interface Semantics}
			\subsubsection{State Variables}
			N/A
			
			\subsubsection{Environmental Variables}
			Not Applicable
			
			\subsubsection{Assumptions}
			{\color{red}Melodies should be initialized before trying to access them}
			
			\subsubsection{Access Program Semantics}
			landpage\_music()
			
			Input: none
			
			Transition: Play melody 0 and 1 on track 0, play melody 2 and 3 on track 1, and play melody 4 on track 2.
			
			Output: none
			
			Exception: none\\
			\\
			game\_music():
			Input: none
			
			Transition: Play melody 2 and 3 on track 1.
			
			Output: none
			Exception: none\\
			\\
			end\_music():
			
			Input: none
			
			Transition: Play melody 4 on track 2.
			
			Output: none
			
			Exception: none\\
			\\
			{\color{red}gift\_music():
			
			Input: none
			
			Transition: Play melody 6 on track 2.
			
			Output: none
			
			Exception: none}\\
			\\
			shoot\_music():
			
			Input: none
			
			Transition: Play melody 5 on track 0.
			
			Output: none
			
			Exception: none\\
			\\
			damage\_music():
			
			Input: none
			
			Transition: Play melody 7 on track 1.
			
			Output: none
			
			Exception: none\\
			\\
			stop\_music():
			
			Input: none
			
			Transition: Stop melodies on all tracks from 0 to 2.
			
			Output: none
			
			Exception: none
			
\section{Major Revision History}
November 7, 2018 - Rough draft of sections\\
November 8, 2018 - Revised sections \\
November 9, 2018 - Revision 0 complete\\
December 5, 2018 - Revision 1 complete\\


			
\end{document}